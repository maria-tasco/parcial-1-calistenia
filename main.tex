\documentclass{article}
\usepackage[utf8]{inputenc}
\usepackage[spanish]{babel}
\usepackage{listings}
\usepackage{graphicx}
\graphicspath{ {images/} }
\usepackage{cite}

\begin{document}

\begin{titlepage}
    \begin{center}
        \vspace*{1cm}
            
        \Huge
        \textbf{Parcial 1}
            
        \vspace{0.5cm}
        \LARGE
        Calistenia
            
        \vspace{1.5cm}
            
        \textbf{Maria Fernanda Tasco Alquichire}
            
        \vfill
            
        \vspace{0.8cm}
            
        \Large
        Despartamento de Ingeniería Electrónica y Telecomunicaciones\\
        Universidad de Antioquia\\
        Medellín\\
        Marzo de 2021
            
    \end{center}
\end{titlepage}

\tableofcontents
\newpage
\section{Introducción}\label{intro}
La calistenia es un sistema de ejercicios físicos con el propio peso corporal. En este sistema, el interés está en los movimientos. En este informe se presenta el desarrollo de un algoritmo para la solución de un problema relacionado con la tecnica de la calistenia, que trata de como llevar unos objetos de una posición A a una posición B. Para el desarrollo de la solución se compartirá la descripción propuesta a 3 personas, con algunas condiciones y se registrará en video.

\section{Sección de contenido} \label{contenido}

\subsection{Algoritmo}
\begin{flushleft}

* Problema: llevar unos objetos de una posición A a una posición B.
    \\[0.2cm]
* Entradas/recursos: 
    \\[0.2cm]
    \begin{flushleft}
        1. Una hoja de papel no tan lisa. \\[0.1cm]
        2. Dos tarjetas del mismo tamaño y peso. \\[0.1cm]
        3. Una superficie horizontal plana.
    \end{flushleft}

* Datos previos o posición inicial: 
    \\[0.2cm]
    \begin{flushleft}
        1. Coloca una tarjeta sobre la otra, sobre la superficie horizontal plana. \\[0.1cm]
        2. Sobre las tarjetas coloca la hoja de papel no tan lisa.
    \end{flushleft}
* Desarrollo  del movimiento:
    \begin{flushleft}
        1. Con una sola mano realizar todos los siguientes movimientos. \\[0.1cm]
        2. Retirar la hoja y colocarla a un lado de la superficie que elejiste para trabajar, en una posición cercana a donde están las tarjetas. \\[0.1cm]
        3. Coger las dos tarjetas y colocarlas juntas verticalmente sobre la hoja sosteniendolas firme con la mano. \\[0.1cm]
        4. Coger una sola tarjeta lentamente y moverla hasta formar un ángulo de 45 grados entre ellas.
        \end{flushleft}
\end{flushleft}        

\subsection{Registro de la solución}
Se compartio el algoritmo definido en la sección anterior a 3 personas a las cuales no se les dio más información que el documento con las intrucciones o sea el algoritmo para que realizaran el ejercicio de forma autonoma; y se registraron los 3 intentos en un video subido posteriormente a \textbf{YouTube} \cite{YouTube}




\section{Conclusiones} \label{conclusiones}
Una clave para el desarrollo del algoritmo fue el orden de los pasos, ya que un orden adecuado, omitir algunos pasos o no generar las condiciones necesarias para el buen desarrollo de la solución, podría generar alguna ambigüedad o error.
\\[0.2cm]
Aprendí la importancia de analizar bien el orden de los pasos, las condiciones, las variables de entrada, lo que me están pidiendo para así generar una solución óptima, sin ambigüedades, eficaz y fácil de entender.

\bibliographystyle{IEEEtran}
\bibliography{references}

\end{document}
